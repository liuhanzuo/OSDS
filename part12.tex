\documentclass{article}
\usepackage{amsmath,amsfonts,amsthm,amssymb}
\usepackage{setspace}
\usepackage{fancyhdr}
\usepackage{lastpage}
\usepackage{extramarks}
\usepackage{chngpage}
\usepackage{soul,color}
\usepackage{graphicx,float,wrapfig}
\usepackage{CJK}
\usepackage{algorithm}
\usepackage{graphicx}
\usepackage{algpseudocode}% http://ctan.org/pkg/algorithmicx

\usepackage{amsthm}

\newtheorem*{lemma}{Lemma}
\newtheorem*{theorem}{Theorem}

\newtheorem*{definition}{Definition}


\newcommand{\Class}{Operating Systems \& Distributive Systems}

% Homework Specific Information. Change it to your own
\newcommand{\Title}{\\Project 1 Part 1.2 Report}
\newcommand{\StudentName}{Ziheng Zhou}
\newcommand{\StudentClass}{}
\newcommand{\StudentNumber}{2024010811}

% In case you need to adjust margins:
\topmargin=-0.45in      %
\evensidemargin=0in     %
\oddsidemargin=0in      %
\textwidth=6.5in        %
\textheight=9.0in       %
\headsep=0.25in         %

% Setup the header and footer
\pagestyle{fancy}                                                       %
\lhead{\StudentName}                                                 %
\chead{\Title}  %
\rhead{\firstxmark}                                                     %
\lfoot{\lastxmark}                                                      %
\cfoot{}                                                                %
\rfoot{Page\ \thepage\ of\ \protect\pageref{LastPage}}                          %
\renewcommand\headrulewidth{0.4pt}                                      %
\renewcommand\footrulewidth{0.4pt}                                      %

%%%%%%%%%%%%%%%%%%%%%%%%%%%%%%%%%%%%%%%%%%%%%%%%%%%%%%%%%%%%%
% Some tools
\newcommand{\enterProblemHeader}[1]{\nobreak\extramarks{#1}{#1 continued on next page\ldots}\nobreak%
                                    \nobreak\extramarks{#1 (continued)}{#1 continued on next page\ldots}\nobreak}%
\newcommand{\exitProblemHeader}[1]{\nobreak\extramarks{#1 (continued)}{#1 continued on next page\ldots}\nobreak%
                                   \nobreak\extramarks{#1}{}\nobreak}%

\newcommand{\homeworkProblemName}{}%
\newcounter{homeworkProblemCounter}%
\newenvironment{homeworkProblem}[1][Problem \arabic{homeworkProblemCounter}]%
  {\stepcounter{homeworkProblemCounter}%
   \renewcommand{\homeworkProblemName}{#1}%
   \section*{\homeworkProblemName}%
   \enterProblemHeader{\homeworkProblemName}}%
  {\exitProblemHeader{\homeworkProblemName}}%

\newcommand{\homeworkSectionName}{}%
\newlength{\homeworkSectionLabelLength}{}%
\newenvironment{homeworkSection}[1]%
  {% We put this space here to make sure we're not connected to the above.

   \renewcommand{\homeworkSectionName}{#1}%
   \settowidth{\homeworkSectionLabelLength}{\homeworkSectionName}%
   \addtolength{\homeworkSectionLabelLength}{0.25in}%
   \changetext{}{-\homeworkSectionLabelLength}{}{}{}%
   \subsection*{\homeworkSectionName}%
   \enterProblemHeader{\homeworkProblemName\ [\homeworkSectionName]}}%
  {\enterProblemHeader{\homeworkProblemName}%

   % We put the blank space above in order to make sure this margin
   % change doesn't happen too soon.
   \changetext{}{+\homeworkSectionLabelLength}{}{}{}}%

\newcommand{\Answer}{\ \\\textbf{Answer:} }
\newcommand{\Acknowledgement}[1]{\ \\{\bf Acknowledgement:} #1}

%%%%%%%%%%%%%%%%%%%%%%%%%%%%%%%%%%%%%%%%%%%%%%%%%%%%%%%%%%%%%


%%%%%%%%%%%%%%%%%%%%%%%%%%%%%%%%%%%%%%%%%%%%%%%%%%%%%%%%%%%%%
% Make title
\title{\textmd{\bf \Class: \Title}}
\author{\textbf{Xinran Li, Hanzuo Liu, Ziheng Zhou}}
%%%%%%%%%%%%%%%%%%%%%%%%%%%%%%%%%%%%%%%%%%%%%%%%%%%%%%%%%%%%%

\begin{document}
\begin{spacing}{1.1}
\maketitle \thispagestyle{empty}
%\cite{}
%%%%%%%%%%%%%%%%%%%%%%%%%%%%%%%%%%%%%%%%%%%%%%%%%%%%%%%%%%%%%
% Begin edit from here

The implementations can be found in \texttt{multi.go}.


\begin{homeworkProblem}[Throughput Analyzis]

We test the throughputs under different values of working threads \& queue lengths. See Figure 1 \& 2.

\begin{figure}[H]
    \centering
    \includegraphics[width=0.5\linewidth]{fig/resize_throughput_threadnum.png}
    \caption{Throughput vs \# of Threads.}
    \label{fig:enter-label}
    
\end{figure}

\begin{figure}[H]
    \centering
    \includegraphics[width=0.5\linewidth]{fig/resize_throughput_queuelength.png}
    \caption{Throughput vs Queue Length.}
    \label{fig:enter-label}
\end{figure}

We conclude that

\begin{itemize}
    \item The throughput grows nearly linearly with \# of threads until there are 20 threads. After that the performance starts to swing irregularly. We infer that the maximum parallelism that our working server can reach is less than 20x.
    \item There is no significant relationship between queue length \& performance. We infer that this is because the main workload is on FS operations / resizing images. We have to mark that there are relatively large fluctuations in the servers' performance; we are currently re-running experiments for 10 times to deal with it.
\end{itemize}
\end{homeworkProblem}

\begin{homeworkProblem}[Latency Analyzis]
    We plot the latency distribution for the case \# of Threads = Queue Length = 300.

\begin{figure}[H]
    \centering
    \includegraphics[width=0.5\linewidth]{fig/resize_latency_1.png}
    \caption{Resize Latency Distribution, 60 Threads, Queue Length = 1000.}
    \label{fig:enter-label}
\end{figure}

\begin{figure}[H]
    \centering
    \includegraphics[width=0.5\linewidth]{fig/resize_latency_2.png}
    \caption{Resize Latency Distribution, 60 Threads, Queue Length = 100.}
    \label{fig:enter-label}
\end{figure}

\begin{figure}[H]
    \centering
    \includegraphics[width=0.5\linewidth]{fig/resize_latency_3.png}
    \caption{Resize Latency Distribution, 30 Threads, Queue Length = 1000.}
    \label{fig:enter-label}
\end{figure}

(In the above 3 charts largest 1$\%$ of entries are removed to make the plot elegant.)


We conclude that

\begin{itemize}
    \item While most queries are finished within a few milliseconds, there are queries that take up to 0.1 seconds.
    \item Distribution graphs look alike for different settings, but the 30 threads case and the 1000 queue length case are both slightly better than the 60 threads, 1000 queue length case. We infer this is because all settings reach maximum parallel level and the more threads / queue size we set, the more context switches / chances of cache miss occur.
\end{itemize}
    
\end{homeworkProblem}


\end{spacing}
\end{document}
