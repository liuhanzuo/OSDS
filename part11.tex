\documentclass{article}
\usepackage{amsmath,amsfonts,amsthm,amssymb}
\usepackage{setspace}
\usepackage{fancyhdr}
\usepackage{lastpage}
\usepackage{extramarks}
\usepackage{chngpage}
\usepackage{soul,color}
\usepackage{graphicx,float,wrapfig}
\usepackage{CJK}
\usepackage{algorithm}
\usepackage{graphicx}
\usepackage{algpseudocode}% http://ctan.org/pkg/algorithmicx

\usepackage{amsthm}

\newtheorem*{lemma}{Lemma}
\newtheorem*{theorem}{Theorem}

\newtheorem*{definition}{Definition}


\newcommand{\Class}{Operating Systems \& Distributive Systems}

% Homework Specific Information. Change it to your own
\newcommand{\Title}{\\Project 1 Part 1.1 Report}
\newcommand{\StudentName}{Ziheng Zhou}
\newcommand{\StudentClass}{}
\newcommand{\StudentNumber}{2024010811}

% In case you need to adjust margins:
\topmargin=-0.45in      %
\evensidemargin=0in     %
\oddsidemargin=0in      %
\textwidth=6.5in        %
\textheight=9.0in       %
\headsep=0.25in         %

% Setup the header and footer
\pagestyle{fancy}                                                       %
\lhead{\StudentName}                                                 %
\chead{\Title}  %
\rhead{\firstxmark}                                                     %
\lfoot{\lastxmark}                                                      %
\cfoot{}                                                                %
\rfoot{Page\ \thepage\ of\ \protect\pageref{LastPage}}                          %
\renewcommand\headrulewidth{0.4pt}                                      %
\renewcommand\footrulewidth{0.4pt}                                      %

%%%%%%%%%%%%%%%%%%%%%%%%%%%%%%%%%%%%%%%%%%%%%%%%%%%%%%%%%%%%%
% Some tools
\newcommand{\enterProblemHeader}[1]{\nobreak\extramarks{#1}{#1 continued on next page\ldots}\nobreak%
                                    \nobreak\extramarks{#1 (continued)}{#1 continued on next page\ldots}\nobreak}%
\newcommand{\exitProblemHeader}[1]{\nobreak\extramarks{#1 (continued)}{#1 continued on next page\ldots}\nobreak%
                                   \nobreak\extramarks{#1}{}\nobreak}%

\newcommand{\homeworkProblemName}{}%
\newcounter{homeworkProblemCounter}%
\newenvironment{homeworkProblem}[1][Problem \arabic{homeworkProblemCounter}]%
  {\stepcounter{homeworkProblemCounter}%
   \renewcommand{\homeworkProblemName}{#1}%
   \section*{\homeworkProblemName}%
   \enterProblemHeader{\homeworkProblemName}}%
  {\exitProblemHeader{\homeworkProblemName}}%

\newcommand{\homeworkSectionName}{}%
\newlength{\homeworkSectionLabelLength}{}%
\newenvironment{homeworkSection}[1]%
  {% We put this space here to make sure we're not connected to the above.

   \renewcommand{\homeworkSectionName}{#1}%
   \settowidth{\homeworkSectionLabelLength}{\homeworkSectionName}%
   \addtolength{\homeworkSectionLabelLength}{0.25in}%
   \changetext{}{-\homeworkSectionLabelLength}{}{}{}%
   \subsection*{\homeworkSectionName}%
   \enterProblemHeader{\homeworkProblemName\ [\homeworkSectionName]}}%
  {\enterProblemHeader{\homeworkProblemName}%

   % We put the blank space above in order to make sure this margin
   % change doesn't happen too soon.
   \changetext{}{+\homeworkSectionLabelLength}{}{}{}}%

\newcommand{\Answer}{\ \\\textbf{Answer:} }
\newcommand{\Acknowledgement}[1]{\ \\{\bf Acknowledgement:} #1}

%%%%%%%%%%%%%%%%%%%%%%%%%%%%%%%%%%%%%%%%%%%%%%%%%%%%%%%%%%%%%


%%%%%%%%%%%%%%%%%%%%%%%%%%%%%%%%%%%%%%%%%%%%%%%%%%%%%%%%%%%%%
% Make title
\title{\textmd{\bf \Class: \Title}}
\author{\textbf{Xinran Li, Hanzuo Liu, Ziheng Zhou}}
%%%%%%%%%%%%%%%%%%%%%%%%%%%%%%%%%%%%%%%%%%%%%%%%%%%%%%%%%%%%%

\begin{document}
\begin{spacing}{1.1}
\maketitle \thispagestyle{empty}
%\cite{}
%%%%%%%%%%%%%%%%%%%%%%%%%%%%%%%%%%%%%%%%%%%%%%%%%%%%%%%%%%%%%
% Begin edit from here

The implementations can be found in \texttt{sync.go}.


\begin{homeworkProblem}[Throughput Analyzis]

We test the throughputs under different values of working threads \& queue lengths. See Figure 1 \& 2.

\begin{figure}[H]
    \centering
    \includegraphics[width=0.5\linewidth]{fig/throughput_threadnum.png}
    \caption{Throughput vs \# of Threads.}
    \label{fig:enter-label}
    
\end{figure}

\begin{figure}[H]
    \centering
    \includegraphics[width=0.5\linewidth]{fig/throughput_queuelength.png}
    \caption{Throughput vs Queue Length.}
    \label{fig:enter-label}
\end{figure}

We conclude that

\begin{itemize}
    \item In general throughput grows when there are more threads, but the growth is sublinear. The reason can be that there are more context switches.
    \item When the queue is of size 1 the throughput is $\bold{surprisingly\ high}$, even higher than the case when the queue is of maximum capable length. We do not know the specific reason. For the rest cases the throughput grows slowly with queue length (the shape of the curve looks like logarithm).
\end{itemize}
\end{homeworkProblem}

\begin{homeworkProblem}[Latency Analyzis]
    We plot the latency distribution for the case \# of Threads = Queue Length = 300.

\begin{figure}[H]
    \centering
    \includegraphics[width=0.5\linewidth]{fig/deque_latency.png}
    \caption{Dequeue Latency Distribution.}
    \label{fig:enter-label}
\end{figure}

\begin{figure}[H]
    \centering
    \includegraphics[width=0.5\linewidth]{fig/enque_latency.png}
    \caption{Enqueue Latency Distribution.}
    \label{fig:enter-label}
\end{figure}

We conclude that

\begin{itemize}
    \item Most dequeue operations finish within 600ns, with density focused on 400ns; enqueue operations are faster with two density peaks at 250ns \& 450ns.
    \item Both types of operations have data entries with over 10000ns of latency. This is due to long block in the queue.
\end{itemize}
    
\end{homeworkProblem}


\end{spacing}
\end{document}
